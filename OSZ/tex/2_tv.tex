\section{Teilversuch 2: Messen einer Amplitude}
	Aus dem Protokoll ist die Amplitude des Signals wie folgt gemessen:
	\begin{center}
		\begin{tabular}{lrr}
			\toprule
			Gerät & $V_\text{max}$ & $V_\text{eff}$\\
			\midrule
			Multimeter & \SI{4.78(6)}{\volt} & \SI{3.384(4)}{\volt}\\
			Oszilloskop & \SI{4.755(5)}{\volt} & \SI{3.376(4)}{\volt}\\
			\bottomrule
		\end{tabular}
	\end{center}
	wobei $V_\text{max} = \sqrt{2}V_\text{eff}$.

	Die Fehlerintervalle der beiden Werten von $V_\text{max}$ und $V_\text{eff}$ überschneidet sich. Folglich stimmen die Werten miteinander überein. Also sind die beiden Messmethode gleichwertig, wenn man die Amplitude eines Signals bestimmen will.


