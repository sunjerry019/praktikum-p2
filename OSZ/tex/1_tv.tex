\section{Teilversuch 1: Basisbedienelemente des Oszilloskops}
	Die Positionseinstellungen \texttt{[Y-POS.I]} und \texttt{[Y-POS.II]} verschiebt die Kurve vertikal im Bildschirm des Oszilloskops. Die Positionseinstellungen \texttt{[X-POS.]} verschiebt die Kurve horizontal im Bildschirm des Oszilloskops. 

	Die Ablenkfaktoren \texttt{[VOLTS/DIV.]} bzw. \texttt{[TIME/DIV.]} vergrößert und verkleinert die dargestellte Kurve in die vertikale bzw. horizontale Richtung. Ob es verkleinert oder vergrößert werden, kommt darauf an, in welcher Richtung sie gedreht sind. 

	Mit diesen zwei Einstellungen kann man die Kurve auf dem Bildschirm beliebig darstellen. Mit Hilfe von \texttt{AUTOSET} sind diese Einstellungen automatisch gestellt, sodass man leicht eine vernünftige Kurve erhaltet. 

	Der Trigger ist eine Einstellung für die Spannungswert, an dem das "Sweep" (Zeitablenkung) anfängt. Damit kann man ein periodisches Signal statisch im Bildschirm darstellen, sodass Messungen gemacht werden kann. Mit dem \texttt{[Level]} Knopf kann man diesen Spannungswert ändern, sodass die Kurve am verschiedene Zeitpunkten anfängt. Man kann auch damit die Aufnahme bei einem ganz bestimmten Punkt eines nicht-periodischen Signals anfangen, was sehr hilfreich sein kann. 