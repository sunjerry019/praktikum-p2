\section{Teilversuch 4: Induktion durch ein zeitlich veränderliches Magnetfeld}
	Aus dem Ohmschen Gesetz gilt:
	\begin{align}
		U = I\!R  && \Leftrightarrow && I = \frac{U}{R}
	\end{align}
	Im Experiment haben wir als Widerstand \SI{0.150(2)}{\ohm}, somit skalieren wir die Abszisse mit $1/\SI{0.150}{\ohm}$:
	\begin{align}
		y-\text{Skala} = \frac{\SI{20}{\milli\volt\per\centi\meter}}{\SI{0.150}{\ohm}} = 133\frac{1}{3} \si{\milli\ampere\per\centi\meter} = \frac{\SI{400}{\milli\ampere}}{\SI{3}{\centi\meter}}
	\end{align}
	Aus dem Schreiberdiagramme 
	\begin{equation*}
		\begin{tabu}{l|ll|ll}
			\toprule
			\text{Kurve} & \text{Punkt } 1 & \text{Punkt } 2 & \text{Steigung}/\si{\milli\ampere\per\second} & U_{\text{ind}} \\
			\midrule
			\Circled{1} & (9,30 , 940,0) & (20,60 , -820,0) & -155,8 & 110,0\\
			\Circled{2} & (8,15 , 840,0) & (22,00 , -840,0) & -121,3 & 84,0\\
			\Circled{3} & (7,85 , 720,0) & (22,60 , -740,0) & -98,98 & 70,0\\
			\Circled{4} & (5,35 , 780,0) & (23,70 , -700,0) & -80,65 & 54,0\\
			\Circled{5} & (4,30 , 500,0) & (24,50 , -440,0) & -46,53 & 32,0\\
			\bottomrule
		\end{tabu}
	\end{equation*}
	\includepdf{./scans/tv4.pdf}
	% Zur Induktionsspannung:
	% \begin{equation*}
	% 	\begin{tabu}{l *{5}{l}}
	% 		\toprule
	% 		\text{Kurve} & \Circled{1} & \Circled{2} & \Circled{3} & \Circled{4} & \Circled{5}\\
	% 		\midrule
	% 		U_{\text{ind}}/\si{\milli\volt} &  \\
	% 		\bottomrule
	% 	\end{tabu}
	% \end{equation*}
	Aus dem Induktionsgesetz gilt:
	\begin{align}
		U_{\text{ind}} &= - N \dv{}{t}\left(\vec{B} \cdot \vec{A}\,\right) = -N\!A \dv{B}{t} = -N\!A\mu_0\left(\frac{4}{5}\right)^{\nicefrac{3}{2}} \frac{N}{r}\cdot\dv{I}{t} \notag \\
		\Rightarrow U_{\text{ind}} &= \left[-\left(\frac{4}{5}\right)^{\nicefrac{3}{2}} \frac{2\mu_0N^2A}{R}\right]\cdot\dv{I}{t}
	\end{align}
	Also ist $U_{\text{ind}}$ proportional zu $\displaystyle \dv{I}{t}$, mit $R = $ Durchmesser wie im Teilversuch $3$.

	Die induzierte Spannung $U_{\text{ind}}$ wurde dann gegen die Steigung $\displaystyle \dv{I}{t}$ im \gnuplot{} geplottet und eine Kurveanpassung zur $\displaystyle U_{\text{ind}} = m\dv{I}{t} + c$ durchgeführt. Der Fehler der jeweiligen Punkten sind hier nicht berücksichtigt:
	\begin{figure}[H]
		\centering
		\input{./plots/tv4-plot.tex}
		\caption{\centering Überprüfung des Induktionsgesetzes \captionbr $\chi^2_{\text{red}} = \num{1.89758} \implies$ Verträgliche Anpassung}
		\label{fig:tvfour-plot}
		\vspace{-1em}
	\end{figure}
	Als Endergebnis erhalten wir:
	\begin{equation*}
		\begin{tabu}{lll}
			\toprule
			\text{Variable} & \text{Wert} & \text{Gerundet} \\
			\midrule
			m & \SI{-64.519(1445)}{\degree} & \SI{-64.5(15)}{\degree}\\
			c & \SI{-0.140(1145)}{\degree} & \SI{-0.1(12)}{\degree} \\
			\bottomrule
		\end{tabu}
	\end{equation*} 