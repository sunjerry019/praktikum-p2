\section{\gnuplot{} Quellcode zur Auswertung von Teilversuch 1}
    \subsection{Galvanische Zelle}
    \label{appdx:gnuplottv1}
    {  
        % % Surpress "errors" in minted
        % https://tex.stackexchange.com/a/289068
        \renewcommand{\fcolorbox}[4][]{#4}
        \begin{minted}[linenos,breaklines,autogobble,frame=leftline,framesep=10pt]{gnuplot}
#!/usr/bin/env gnuplot

set term epslatex color size 6in, 4in
set output "tv1-plot.tex"
set decimalsign locale 'de_DE.UTF-8'

set title "Klemmenspannung einer galvanischen Zelle gegen den Belastungsstrom"
set ylabel "Spannung $U$ ($\\si{\\volt}$)"
set xlabel "Belastungsstrom $I$ ($\\si{\\milli\\ampere}$)"

set mxtics
set mytics
set samples 10000

f(x) = m*x + c

# (x, y, xdelta, ydelta)
fit f(x) "tv1.dat" u 1:2:(0.008*$1 + 0.001):(0.005*$2 + 0.001) xyerrors via m,c

set xrange [0:40]

# Linien
set key top right spacing 1.3

titel = "$".gprintf("%.5f", m)."I + ".gprintf("%.5f", c)."$"
plot f(x) title titel lc rgb 'dark-magenta', \
    "tv1.dat" u 1:2:(0.008*$1 + 0.001):(0.005*$2 + 0.001) with xyerrorbars title "Messpunkte" pointtype 0 lc rgb 'dark-goldenrod'
        \end{minted}
    }
    mit \texttt{tv1.dat}:
    \begin{multicols}{2}
        \begin{minted}[linenos,breaklines,autogobble,frame=leftline,framesep=10pt]{text}
# I/mA  U/V
10,6    1,332
13,4    1,322
16,2    1,317
19,1    1,312
21,9    1,308
24,8    1,303
27,5    1,300
30,6    1,294
33,1    1,290
35,9    1,286
38,7    1,282
        \end{minted}
    \end{multicols}
    \vspace{-\baselineskip}
    Rohausgabe:
    \begin{minted}[linenos,breaklines,autogobble,frame=leftline,framesep=10pt]{text}
final sum of squares of residuals : 0.472523
rel. change during last iteration : -1.94302e-08

degrees of freedom    (FIT_NDF)                        : 9
rms of residuals      (FIT_STDFIT) = sqrt(WSSR/ndf)    : 0.229134
variance of residuals (reduced chisquare) = WSSR/ndf   : 0.0525026
p-value of the Chisq distribution (FIT_P)              : 0.999976

Final set of parameters            Asymptotic Standard Error
=======================            ==========================
m               = -0.00167333      +/- 5.845e-05    (3.493%)
c               = 1.34552          +/- 0.001544     (0.1148%)

correlation matrix of the fit parameters:
                m      c      
m               1.000 
c              -0.942  1.000
    \end{minted}

    \subsection{Netzgerät}
    {  
        % % Surpress "errors" in minted
        % https://tex.stackexchange.com/a/289068
        \renewcommand{\fcolorbox}[4][]{#4}
        \begin{minted}[linenos,breaklines,autogobble,frame=leftline,framesep=10pt]{gnuplot}
#!/usr/bin/env gnuplot

set term epslatex color size 6in, 4in
set output "tv1-ng-plot.tex"
set decimalsign locale 'de_DE.UTF-8'

set title "Klemmenspannung eines Netzgeräts gegen den Belastungsstrom"
set ylabel "Spannung $U$ ($\\si{\\volt}$)"
set xlabel "Belastungsstrom $I$ ($\\si{\\milli\\ampere}$)"

set mxtics
set mytics
set samples 10000

set style data lines

# Linien
set key top right spacing 1.3

plot "tv1-ng.dat" u 1:2:3:(0.005*$2 + 0.001) with xyerrorbars title "Messpunkte" pointtype 0 lc rgb 'dark-goldenrod', \
     '' using 1:2 smooth mcspline lw 2 lc rgb 'dark-magenta' title "glatter Spline durch Daten"
        \end{minted}
    }
    mit \texttt{tv1-ng.dat}:
    \begin{multicols}{2}
        \begin{minted}[linenos,breaklines,autogobble,frame=leftline,framesep=10pt]{text}
#I/mA   U/V Delta I/mA
104,2   9,99    0,9336
113,4   9,99    1,0072
114,6   9,99    1,0168
280     9,97    3,24
381     9,96    4,048
506     9,95    5,048
619     9,94    5,952
646     9,93    6,168
796     9,92    7,368
898     9,91    8,184
957     9,90    8,656
1106    9,88    9,848
1334    9,32    11,672
1427    7,68    10,016
        \end{minted}
    \end{multicols}
\section{\gnuplot{} Quellcode zur Auswertung von Teilversuch 2}
    \label{appdx:gnuplottv2}
    {  
        % % Surpress "errors" in minted
        % https://tex.stackexchange.com/a/289068
        \renewcommand{\fcolorbox}[4][]{#4}
        \begin{minted}[linenos,breaklines,autogobble,frame=leftline,framesep=10pt]{gnuplot}
#!/usr/bin/env gnuplot

set term epslatex color size 6in, 4in
set output "tv2-plot.tex"
set decimalsign locale 'de_DE.UTF-8'

set title "Abhängigkeit von Strom und Spannung"
set ylabel "Spannung $U$ ($\\si{\\volt}$)"
set xlabel "Strom $I$ ($\\si{\\milli\\ampere}$)"

set mxtics
set mytics
set samples 10000

f(x) = m*x + c

# (x, y, xdelta, ydelta)
fit f(x) "tv2.dat" u 2:1:(0.008*$2 + 0.1):(0.005*$1 + 0.01) xyerrors via m,c

# Linien
set key top left Left spacing 1.3

titel = "$".gprintf("%.5f", m)."I + ".gprintf("%.5f", c)."$"
plot f(x) title titel lc rgb 'dark-magenta', \
    "tv2.dat" u 2:1:(0.008*$2 + 0.1):(0.005*$1 + 0.01) with xyerrorbars title "Messpunkte" pointtype 0 lc rgb 'dark-goldenrod'
        \end{minted}
    }
    mit \texttt{tv2.dat}:
    \begin{multicols}{2}
        \begin{minted}[linenos,breaklines,autogobble,frame=leftline,framesep=10pt]{text}
# U/V   I/mA
2,07    0,6
4,08    1,2
6,01    1,8
8,02    2,4
10,06   3,0
12,15   3,6
14,14   4,3
16,09   4,8
18,02   5,4
19,99   6,0
        \end{minted}
    \end{multicols}
    \vspace{-\baselineskip}
    Rohausgabe:
    \begin{minted}[linenos,breaklines,autogobble,frame=leftline,framesep=10pt]{text}
final sum of squares of residuals : 0.348685
rel. change during last iteration : -1.61199e-09

degrees of freedom    (FIT_NDF)                        : 8
rms of residuals      (FIT_STDFIT) = sqrt(WSSR/ndf)    : 0.208772
variance of residuals (reduced chisquare) = WSSR/ndf   : 0.0435856
p-value of the Chisq distribution (FIT_P)              : 0.999967

Final set of parameters            Asymptotic Standard Error
=======================            ==========================
m               = 3.31938          +/- 0.0161       (0.485%)
c               = 0.0757801        +/- 0.05424      (71.58%)

correlation matrix of the fit parameters:
                m      c      
m               1.000 
c              -0.862  1.000 
    \end{minted}
\section{\gnuplot{} Quellcode zur Auswertung von Teilversuch 3}
    \label{appdx:gnuplottv3}
    {  
        % % Surpress "errors" in minted
        % https://tex.stackexchange.com/a/289068
        \renewcommand{\fcolorbox}[4][]{#4}
        \begin{minted}[linenos,breaklines,autogobble,frame=leftline,framesep=10pt]{gnuplot}
#!/usr/bin/env gnuplot

set term epslatex color size 6in, 4in
set output "tv3-plot.tex"
set decimalsign locale 'de_DE.UTF-8'

set title "Ausgangsspannung des Helipots gegen Skalenwerte des Helipots"
set ylabel "Spannung $U$ ($\\si{\\volt}$)"
set xlabel "Skala $x$ (Schritt)"

set mxtics
set mytics
set samples 10000

f(x) = m*x + c

# (x, y, xdelta, ydelta)
fit f(x) "tv3.dat" u 1:2:(0.5):(0.005*$2 + 0.01) xyerrors via m,c

# Linien
set key bottom right spacing 1.3

titel = "$".gprintf("%.5f", m)."x + ".gprintf("%.5f", c)."$"
plot f(x) title titel lc rgb 'dark-magenta', \
    "tv3.dat" u 1:2:(0.5):(0.005*$2 + 0.01) with xyerrorbars title "Ausgangsspannung des Helipots" pointtype 0 lc rgb 'dark-goldenrod', \
    "tv3.dat" u 1:3:(0.5):(0.005*$2 + 0.01) with xyerrorbars title "Ausgangsspannung des Netzgeräts" pointtype 0 lc rgb 'dark-green'
        \end{minted}
    }
    mit \texttt{tv3.dat}:
    \begin{multicols}{2}
        \begin{minted}[linenos,breaklines,autogobble,frame=leftline,framesep=10pt]{text}
# Skalawert Helipot Netzgerät
1000    10,03   10,03
900     9,02    10,03
800     8,01    10,03
700     7,02    10,03
600     6,01    10,02
500     5,01    10,02
400     4,01    10,02
300     3,01    10,02
200     2,01    10,02
100     1,01    10,02
0       0,01    10,02
        \end{minted}
    \end{multicols}
    \vspace{-\baselineskip}
    Rohausgabe:
    \begin{minted}[linenos,breaklines,autogobble,frame=leftline,framesep=10pt]{text}
final sum of squares of residuals : 0.109204
rel. change during last iteration : -1.71244e-09

degrees of freedom    (FIT_NDF)                        : 9
rms of residuals      (FIT_STDFIT) = sqrt(WSSR/ndf)    : 0.110154
variance of residuals (reduced chisquare) = WSSR/ndf   : 0.0121338
p-value of the Chisq distribution (FIT_P)              : 1

Final set of parameters            Asymptotic Standard Error
=======================            ==========================
m               = 0.0100079        +/- 3.216e-06    (0.03213%)
c               = 0.0091885        +/- 0.001003     (10.91%)

correlation matrix of the fit parameters:
                m      c      
m               1.000 
c              -0.617  1.000 
    \end{minted}