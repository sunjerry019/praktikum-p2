\section{\gnuplot{} Quellcode zur Auswertung von Teilversuch 1}
	{  
        % % Surpress "errors" in minted
        % https://tex.stackexchange.com/a/289068
        \renewcommand{\fcolorbox}[4][]{#4}
        \begin{minted}[linenos,breaklines,autogobble,frame=leftline,framesep=10pt]{gnuplot}
#!/usr/bin/env gnuplot

set term epslatex color size 6in, 4in
set output "tv1-plot.tex"
set decimalsign locale 'de_DE.UTF-8'

set title "Erwärmung von Wasser im Kalorimeter"
set ylabel "Temperatur $\\theta$ ($\\si{\\celsius}$)"
set xlabel "Zeit $t$ ($\\si{\\second}$)"

set mxtics
set mytics
set samples 10000

f(x) = m*x + c

# (x, y, xdelta, ydelta)
fit f(x) "tv1.dat" u 1:2:(0.2):(0.3) xyerrors via m,c

# Linien
set key top left Left spacing 1.3

titel = "$".gprintf("%.5f", m)."t + ".gprintf("%.5f", c)."$"
plot f(x) title titel lc rgb 'dark-magenta', \
	"tv1.dat" u 1:2:(0.2):(0.3) with xyerrorbars title "Messpunkte" pointtype 2 lc rgb 'dark-goldenrod'
        \end{minted}
    }
	mit \texttt{tv1.dat}:
    \begin{multicols}{2}
        \begin{minted}[linenos,breaklines,autogobble,frame=leftline,framesep=10pt]{text}
# t/s  theta/degC
0      25,6
60     26,5
120    26,7
180    27,0
240    27,6
300    28,1
360    28,6
420    29,1
480    29,5
540    30,2
600    30,8
660    31,2
720    31,6
780    32,0
840    32,9
900    33,1
        \end{minted}
    \end{multicols}
    \vspace{-\baselineskip}
	Rohausgabe:
	\begin{minted}[linenos,breaklines,autogobble,frame=leftline,framesep=10pt]{text}
final sum of squares of residuals : 3.32228
rel. change during last iteration : -3.25589e-08

degrees of freedom    (FIT_NDF)                        : 14
rms of residuals      (FIT_STDFIT) = sqrt(WSSR/ndf)    : 0.487141
variance of residuals (reduced chisquare) = WSSR/ndf   : 0.237306
p-value of the Chisq distribution (FIT_P)              : 0.998351

Final set of parameters            Asymptotic Standard Error
=======================            ==========================
m               = 0.00831618       +/- 0.0001321    (1.588%)
c               = 25.664           +/- 0.06977      (0.2719%)

correlation matrix of the fit parameters:
                m      c      
m               1.000 
c              -0.852  1.000 
	\end{minted}

\section{\gnuplot{} Quellcode zur Auswertung von Teilversuch 2}
    {  
        % % Surpress "errors" in minted
        % https://tex.stackexchange.com/a/289068
        \renewcommand{\fcolorbox}[4][]{#4}
        \begin{minted}[linenos,breaklines,autogobble,frame=leftline,framesep=10pt]{gnuplot}
#!/usr/bin/env gnuplot

set term epslatex color size 6in, 4in
set output "tv2-plot.tex"
set decimalsign locale 'de_DE.UTF-8'

set title "Temperaturverlauf vor und nach Eintauchen von Probekörpern"
set ylabel "Temperatur $\\theta$ ($\\si{\\celsius}$)"
set xlabel "Zeit $t$ ($\\si{\\second}$)"

set mxtics
set mytics
set samples 10000

f(x) = m*x + c
g(x) = n*x + d

# (x, y, xdelta, ydelta)
fit [-20:0] f(x) "tv2-1.dat" u 1:2:(0.5):(0.5) xyerrors via m,c
fit [0:400] g(x) "tv2-1.dat" u 1:2:(0.5):(0.5) xyerrors via n,d

titelaone = "1: $".gprintf("%.5f", m)."x + (".gprintf("%.5f", c).")$"
titelatwo = "1: $".gprintf("%.5f", n)."x + (".gprintf("%.5f", d).")$"

h(x) = p*x + o
q(x) = j*x + k

fit [-20:0] h(x) "tv2-2.dat" u 1:2:(0.5):(0.5) xyerrors via p,o
fit [0:400] q(x) "tv2-2.dat" u 1:2:(0.5):(0.5) xyerrors via j,k

titelbone = "2: $".gprintf("%.5f", p)."x + (".gprintf("%.5f", o).")$"
titelbtwo = "2: $".gprintf("%.5f", j)."x + (".gprintf("%.5f", k).")$"

# Linien
set key bottom right spacing 1.3

set yrange [22:34]

plot f(x) title titelaone, \
     g(x) title titelatwo, \
     h(x) title titelbone, \
     q(x) title titelbtwo, \
    "tv2-1.dat" u 1:2:(0.5):(0.5) with xyerrorbars title "Körper 1" pointtype 0, \
    "tv2-2.dat" u 1:2:(0.5):(0.5) with xyerrorbars title "Körper 2" pointtype 0
        \end{minted}
    }
    mit \texttt{tv2-1.dat}:
    \begin{multicols}{3}
        \begin{minted}[linenos,breaklines,autogobble,frame=leftline,framesep=10pt]{text}
# t/s   Theta/deg C
-20 26,5
-15 26,2
-10 26,3
-5  26,3
5   32,3
10  32,8
15  32,4
20  32,3
25  31,8
30  31,6
35  32,2
40  31,2
90  31,6
120 31,7
150 32,0
193 32,1
232 31,6
240 31,6
270 31,1
300 31,7
335 31,5
370 31,2
        \end{minted}
    \end{multicols}
    und \texttt{tv2-2.dat}:
    \begin{multicols}{3}
        \begin{minted}[linenos,breaklines,autogobble,frame=leftline,framesep=10pt]{text}
# t/s   Theta/deg C
-20 25,0
-15 24,5
-10 24,2
-5  24,7
2   27,2
5   30,2
10  29,1
15  28,7
20  28,4
25  28,5
30  28,5
35  28,5
40  28,4
45  28,5
75  28,3
105 28,2
135 28,1
165 28,4
195 28,3
225 28,4
255 28,3
285 28,2
315 28,4
        \end{minted}
    \end{multicols}

    Rohausgabe:
    \begin{minted}[linenos,breaklines,autogobble,frame=leftline,framesep=10pt]{text}
final sum of squares of residuals : 0.139986
rel. change during last iteration : 0

degrees of freedom    (FIT_NDF)                        : 2
rms of residuals      (FIT_STDFIT) = sqrt(WSSR/ndf)    : 0.264562
variance of residuals (reduced chisquare) = WSSR/ndf   : 0.069993
p-value of the Chisq distribution (FIT_P)              : 0.9324

Final set of parameters            Asymptotic Standard Error
=======================            ==========================
m               = -0.0100012       +/- 0.01183      (118.3%)
c               = 26.2             +/- 0.162        (0.6184%)

correlation matrix of the fit parameters:
                m      c      
m               1.000 
c               0.913  1.000 

-----------------------------------------------------------------------

final sum of squares of residuals : 9.08037
rel. change during last iteration : -5.8453e-13

degrees of freedom    (FIT_NDF)                        : 16
rms of residuals      (FIT_STDFIT) = sqrt(WSSR/ndf)    : 0.753341
variance of residuals (reduced chisquare) = WSSR/ndf   : 0.567523
p-value of the Chisq distribution (FIT_P)              : 0.910067

Final set of parameters            Asymptotic Standard Error
=======================            ==========================
n               = -0.00224382      +/- 0.0007261    (32.36%)
d               = 32.1258          +/- 0.1338       (0.4163%)

correlation matrix of the fit parameters:
                n      d      
n               1.000 
d              -0.748  1.000 

-----------------------------------------------------------------------

final sum of squares of residuals : 1.07138
rel. change during last iteration : 0

degrees of freedom    (FIT_NDF)                        : 2
rms of residuals      (FIT_STDFIT) = sqrt(WSSR/ndf)    : 0.731909
variance of residuals (reduced chisquare) = WSSR/ndf   : 0.535691
p-value of the Chisq distribution (FIT_P)              : 0.585264

Final set of parameters            Asymptotic Standard Error
=======================            ==========================
p               = -0.024001        +/- 0.03274      (136.4%)
o               = 24.3             +/- 0.4483       (1.845%)

correlation matrix of the fit parameters:
                p      o      
p               1.000 
o               0.913  1.000 

-----------------------------------------------------------------------

final sum of squares of residuals : 20.581
rel. change during last iteration : -9.95489e-07

degrees of freedom    (FIT_NDF)                        : 17
rms of residuals      (FIT_STDFIT) = sqrt(WSSR/ndf)    : 1.10029
variance of residuals (reduced chisquare) = WSSR/ndf   : 1.21064
p-value of the Chisq distribution (FIT_P)              : 0.245597

Final set of parameters            Asymptotic Standard Error
=======================            ==========================
j               = -0.00123965      +/- 0.001241     (100.1%)
k               = 28.5819          +/- 0.1808       (0.6325%)

correlation matrix of the fit parameters:
                j      k      
j               1.000 
k              -0.716  1.000 
    \end{minted}

\section{\gnuplot{} Quellcode zur Auswertung von Teilversuch 5}
    \label{appdx:gnuplotTV5}
    {  
        % % Surpress "errors" in minted
        % https://tex.stackexchange.com/a/289068
        \renewcommand{\fcolorbox}[4][]{#4}
        \begin{minted}[linenos,breaklines,autogobble,frame=leftline,framesep=10pt]{gnuplot}
#!/usr/bin/env gnuplot

set term epslatex color size 6in, 4in
set output "tv5-plot.tex"
set decimalsign locale 'de_DE.UTF-8'

set title "Strahlung eines Hohlraumstrahlers"
set ylabel "Thermopannung $V$ ($\\si{\\micro\\volt}$)"
set xlabel "Temperatur $(\\theta^4 - T_0^4)$ ($\\times 10^{10} \\si{\\kelvin^4}$)"

set mxtics
set mytics
set samples 10000

f(x) = m*x + c

# (x, y, xdelta, ydelta)
fit f(x) "tv5.dat" u ((($1 + 273.15)**4 - (29+273.15)**4)/10**10):2:((4*0.1*sqrt(($1+273.15)**6 + (29+273.15)**6))/10**10):(2) xyerrors via m,c

# Linien
set key top left Left spacing 1.3

titel = "$".gprintf("%.5f", m)."x + (".gprintf("%.5f", c).")$"
plot f(x) title titel lc rgb 'dark-magenta', \
    "tv5.dat" u ((($1 + 273.15)**4 - (29+273.15)**4)/10**10):2:((4*0.1*sqrt(($1+273.15)**6 + (29+273.15)**6))/10**10):(2) with xyerrorbars title "Messpunkte" pointtype 0 lc rgb 'dark-goldenrod'
        \end{minted}
    }
    mit \texttt{tv5.dat}:
    \begin{minted}[linenos,breaklines,autogobble,frame=leftline,framesep=10pt]{text}
#T/C Spannung
80   10
100  18
130  25
160  52
190  63
210  79
240 108
270 139
300 174
330 222
350 250
    \end{minted}
    % \vspace{-\baselineskip}
Rohausgabe:
    \begin{minted}[linenos,breaklines,autogobble,frame=leftline,framesep=10pt]{text}
final sum of squares of residuals : 21.9449
rel. change during last iteration : -7.37544e-09

degrees of freedom    (FIT_NDF)                        : 9
rms of residuals      (FIT_STDFIT) = sqrt(WSSR/ndf)    : 1.56151
variance of residuals (reduced chisquare) = WSSR/ndf   : 2.43832
p-value of the Chisq distribution (FIT_P)              : 0.00905537

Final set of parameters            Asymptotic Standard Error
=======================            ==========================
m               = 17.8711          +/- 0.2133       (1.193%)
c               = -2.41982         +/- 1.577        (65.15%)

correlation matrix of the fit parameters:
                m      c      
m               1.000 
c              -0.801  1.000 
    \end{minted}